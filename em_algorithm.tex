\documentclass[11pt]{article}

% ---------------- Packages ---------------------------
\usepackage[usenames,dvipsnames]{color} % for adding color
\usepackage{amsmath, amsthm, amssymb}   % for maths
\usepackage[in,myheadings]{fullpage}    % for setting margins
\usepackage{enumerate}  % for setting the enumerate bullet style
\usepackage{fancyhdr}   % for setting page headers and footers
\usepackage{listings}   % for typesetting code
\usepackage{graphicx}   % for 'enhanced' graphics package
\usepackage{framed}     % for frame borders
\usepackage{float}      % for absolute figure placement (H)
\usepackage{url}        % for typesetting hyperlinks
\usepackage{bm}         % for bold greek letters


% ---------------- Environments -----------------------
\newenvironment{pitemize}{
    \vspace{-3mm}
    \begin{itemize}
        \setlength{\itemsep}{1pt}
        \setlength{\parskip}{0pt}
        \setlength{\parsep}{0pt}
}{\end{itemize}}

\newenvironment{penumerate}{
    \begin{enumerate}
        \vspace{-3mm}
        \setlength{\itemsep}{1pt}
        \setlength{\parskip}{-2pt}
        \setlength{\parsep}{-5pt}
}{\end{enumerate}}

\newenvironment{definition}[1][Definition]{
    \begin{trivlist}
    \item[\hskip \labelsep {\bfseries #1}]
}{\end{trivlist}}

\newenvironment{example}[1][Example]{
    \begin{trivlist}
    \item[\hskip \labelsep {\bfseries #1}]
}{\end{trivlist}}

\newenvironment{summary}[1][Summary]{
    \begin{trivlist}
    \item[\hskip \labelsep {\bfseries #1}]
}{\end{trivlist}}

% ---------------- General Macros --------------------
\newcommand{\ul}[1]{\underline{#1}}             % underline
\newcommand{\todo}[1]{\textcolor{red}{\textbf{#1}}}  % todo
\newcommand{\qpart}[1]{\subsection{#1}\vspace{-3mm}}
\newcommand{\qsection}[1]{\section{#1}\vspace{-3mm}}

% ---------------- Probability Macros -----------------
\renewcommand{\L}[1]{\mathcal{L}}    % likelihood
\newcommand{\E}[1]{\mathbb{E}}     % expectation
\newcommand{\Cov}[1]{\text{Cov}}   % covariance
\newcommand{\Var}[1]{\text{Var}}   % variance
\newcommand{\where}[1]{\text{ where }}
\newcommand{\subto}[1]{\text{ subject to }}

% ---------------- Linear Algebra Macros --------------
\newcommand{\bs}{\boldsymbol}      % bold greek letters
\newcommand{\norm}[1]{\left\lVert#1\right\rVert} % norm

% ---------------- Number Theory Macros ---------------
\newcommand{\R}{\mathbb{R}}
\newcommand{\N}{\mathcal{N}}

% ---------------- Document Settings ------------------
\setlength{\headheight}{20pt}
\setlength{\textheight}{8in}
\setlength{\footskip}{0.8in}
\setlength{\parskip}{6pt plus 2pt minus 1pt}
\setlength{\parindent}{0pt}
\pagestyle{fancy}

% Headers
\lhead{\\EM Algorithm Primer }
\rhead{\\
David Bourgin}

\begin{document}
The \textbf{EM algorithm} is a method for fitting latent variable models. Let:
\begin{pitemize}
    \item $X = \{ x_1, \ldots, x_N \}$ be the collection of observed data points
    \item $\theta$ be a collection of model parameters
    \item $T = \{t_1, \ldots t_N \}$ be the collection of latent variables
        associated with each data point
\end{pitemize}

\section*{Variational Lower Bound on Marginal Likelihood}
In fitting our model we will attempt to find the setting of the parameters
$\theta$ that maximizes the \textbf{marginal likelihood} of the data:
\begin{align*}
    P(X|\theta) &= \prod_{i=1}^N P(x_i | \theta) &&\text{Assume data are iid}\\
    % &= \prod_{i=1}^N \sum_{c=1}^T P(x_i | t_i = c, \theta) P(t_i = c |
    % \theta) \\
    &= \prod_{i=1}^N \sum_{c=1}^T P(x_i, t_i = c | \theta)
\end{align*}
As always, it is generally easier to maximize the \textbf{marginal log
likelihood} rather than the likelihood directly:
\begin{align*}
    \log P(X|\theta) &= \log \prod_{i=1}^N P(x_i | \theta) &&\text{Assume data
    are iid.}\\
    &= \sum_{i=1}^N \log P(x_i | \theta)\\
    &= \sum_{i=1}^N \log \left[ \sum_{c=1}^T P(x_i, t_i = c | \theta )\right]
\end{align*}
The problem is that this expression is still difficult to optimize directly
(e.g., via SGD). In EM, we opt to instead try to maximize a \textbf{lower
bound}, $\mathcal{L}$ on the marginal log likelihood instead:
\begin{equation*}
    \overbrace{\log P(X|\theta)}^{\text{Marginal log likelihood}} \geq \underbrace{\mathcal{L}}_{\text{Variational lower
    bound}}
\end{equation*}
The issue is that there is no reason to expect a single lower bound to be useful
for finding a local maxima of the marginal log likelihood. What we really want
is a  \emph{family} of lower bounds, which we can tune to get better and better 
local approximations to the marginal log likelihood at $\theta$. To achieve this, we introduce a
new parameter to the lower bound, a distribution over the latent variable classes:
\begin{equation*}
    q(t_i = c)
\end{equation*}
This distribution will be used as a flexible parameter of our family of lower
bounds, $\mathcal{L}$, allowing us modify the form of the lower bound  over the
course of optimization.\\
\\
We derive the form for the family of lower bounds using \textbf{Jensen's
inequality}:
\begin{equation}
    \log (\mathbb{E}[X]) \geq \mathbb{E}[\log X]
\end{equation}
or, if we assume $X$ is a discrete random variable:
\begin{equation}
    \log \sum_i \alpha_i x_i \geq \sum_i \alpha_i \log
    x_i
\end{equation}
where $\alpha_i \geq 0 \ \forall i$ and $\sum_i \alpha_i = 1$. Using this inequality, we can derive a lower bound
on the marginal log likelihood:
\begin{align*}
    \log P(X|\theta) &= \sum_{i=1}^N \log \left[ \sum_{c=1}^T 
    P(x_i, t_i = c | \theta) \right]\\
    &= \sum_{i=1}^N \log \left[ \sum_{c=1}^T \underbrace{\frac{q(t_i = c)}{q(t_i
    = c)}}_{\text{this is just 1}} \times
    P(x_i, t_i = c | \theta) \right]\\
\end{align*}
At this point, notice that we can rewrite the last line as 
\begin{equation*}
    \log P(X|\theta) &= \sum_{i=1}^N \log \mathbb{E}_q \left[\frac{P(x_i, T |
    \theta)}{q(T)} \right]
\end{equation*}
This allows us to apply Jensen's inequality (Eq. 1), to define a family of lower bounds:
\begin{align*}
    \log P(X|\theta) &\geq \mathcal{L}(\theta, q)\\
    \sum_{i=1}^N \log \mathbb{E}_q \left[\frac{P(x_i, T | \theta)}{q(T)} \right]
    &\geq \sum_{i=1}^N \mathbb{E}_q \left[ \log \frac{P(x_i, T | \theta)}{q(T)}
    \right]\\
    \sum_{i=1}^N \log \left[ \sum_{c=1}^T \frac{q(t_i = c)}{q(t_i
    = c)} \times P(x_i, t_i = c | \theta) \right] &\geq \underbrace{\sum_{i=1}^N \sum_{c=1}^T
    q(t_i = c) \log \frac{P(x_i, t_i = c | \theta)}{q(t_i =
    c)}}_{\mathcal{L}(\theta, q)}\\
\end{align*}

\begin{framed}
    \vspace{-5mm}
    \begin{summary} \textit{Variational Lower Bound}\\
        We have now derived a \emph{family} of lower bounds on the marginal log
        likelihood, $\log P(X | \theta)$. The functions in this family are of
        the form
        \begin{align*}
            \mathcal{L}(\theta, q) &= \sum_{i=1}^N \sum_{c=1}^T q(t_i = c) \log
            \left[ \frac{P(x_i, t_i = c | \theta)}{q(t_i = c)} \right]\\
            &= \sum_{i=1}^N \mathbb{E}_{q} \left[\log \frac{P(x_i, T | \theta)}{q(T)}\right]
        \end{align*}
        During optimization, we can modify the distribution $q$ to achieve
        lower bounds that provide better local approximations to $\log
        P(X|\theta)$ at $\theta$.
    \end{summary}
    \vspace{-2mm}
\end{framed}

\section*{EM Algorithm Overview}
The EM algorithm is an iterative approach to coordinate ascent on the marginal
likelihood, $P(X|\theta)$. It consists of two steps, which are repeated until
convergence:
\begin{enumerate}
    \item \textbf{E-Step}: Given a starting value for $\theta$, find the
        distribution $q^*$ that maximizes $\mathcal{L}(\theta, q)$.
    \item \textbf{M-Step}: Given the distribution $q$ identified during the
        E-step, find the value of $\theta^*$ that maximizes $\mathcal{L}(\theta,
        q^*)$.
\end{enumerate}

\begin{figure}[htpb]
    \centering
    \includegraphics[width=\linewidth]{figs/em.png}
\end{figure}

\subsection*{E-Step Details}
During the E-step, we fix the current value for the parameters, $\theta$, and
try to maximize the variational lower bound, $\mathcal{L}$, with respect to the
distribution $q$:
\begin{equation*}
    q^{(k+1)} = \arg \max_q \mathcal{L}(\theta^{(k)}, q)
\end{equation*}
This maximization problem is equivalent to \emph{minimizing} the gap between $\log
P(X|\theta^{(k)})$ and $\mathcal{L}(\theta^{(k)}, q)$:
\begin{equation*}
    q^{(k+1)} = \arg \min_q \log P(X|\theta^{(k)}) - \mathcal{L}(\theta^{(k)},
    q)
\end{equation*}
We can rewrite the gap between $\log P(X|\theta^{(k)})$ and
$\mathcal{L}(\theta^{(k)}, q)$ as:
\begin{align*}
    &\log P(X|\theta) - \mathcal{L}(\theta, q)\\
    &= \sum_{i=1}^N \log P(x_i|\theta)  - \sum_{i=1}^N
    \sum_{c=1}^T q(t_i = c) \log \left[ \frac{P(x_i, t_i = c |
    \theta)}{q(t_i=c)} \right]\\
    &= \sum_{i=1}^N \left( \log P(x_i | \theta) \underbrace{\sum_{c=1}^T q(t_i =
    c)}_{\text{this is just 1}} - \sum_{c=1}^T q(t_i = c) \log \left[ \frac{P(x_i,
    t_i = c | \theta)}{q(t_i = c)} \right] \right)\\
    &= \sum_{i=1}^N \sum_{c=1}^T q(t_i = c) \left( \log P(x_i | \theta) - \log
    \frac{P(x_i, t_i = c | \theta)}{q(t_i = c)} \right)\\
    &= \sum_{i=1}^N \sum_{c=1}^T q(t_i = c) \log \left[ \frac{P(x_i | \theta)
    q(t_i = c)}{P(x_i, t_i = c | \theta) } \right]\\
    &= \sum_{i=1}^N \sum_{c=1}^T q(t_i = c) \log \left[ \frac{P(x_i | \theta)
    q(t_i = c)}{P(t_i = c | x_i, \theta)  P(x_i | \theta)} \right]\\
    &= \sum_{i=1}^N \sum_{c=1}^T q(t_i=c) \log \left( \frac{q(t_i =
    c)}{P(t_i=c|x_i, \theta)} \right) \\
    &= \sum_{i=1}^N \mathbb{KL}(q(t_i) \ || \ P(t_i| x_i, \theta))
\end{align*}
Thus we have that during the E-step, 
\begin{align*}
    q^{(k+1)} &= \arg \min_q \log P(X|\theta^{(k)}) - \mathcal{L}(\theta^{(k)},
    q)\\
    &= \arg \min_q \sum_{i=1}^N \mathbb{KL}(q(t_i) \ || \ P(t_i| x_i, \theta))
\end{align*}
Because the smallest KL-divergence is achieved when $q(t_i) = P(t_i| x_i,
\theta)$, we conclude that the update for the \textbf{E-step} is simply:
\begin{equation}
    q^{(k+1)} = \underbrace{P(T | X, \theta^{(k)})}_{\text{Posterior over
    latent classes}}
\end{equation}

\begin{framed}
    \vspace{-5mm}
    \begin{summary} \textit{E-Step}\\
        During the \textbf{E-step}, we fix the current value for the parameters, $\theta$, and
        try to maximize the variational lower bound, $\mathcal{L}$, with respect to the
        distribution $q$.\\
        \\
        Above, we demonstrate that maximizing $\mathcal{L}$ wrt $q$ is
        equivalent to minimizing the gap between $\log P(X|\theta)$ and
        $\mathcal{L}$, which is in turn equivalent to minimizing the sum of the
        KL divergences between $q(t_i)$ and $P(t_i | x_i, \theta)$.\\
        \\
        This observation implies that the update during the \textbf{E-step}
        should simply be:
        \begin{equation*}
            q^{(k+1)} = P(T | X, \theta^{(k)})
        \end{equation*}
        The caveat is that often it is intractable to compute the posterior
        over latent classes, $P(T | X, \theta)$, exactly. In these cases, it is necessary
        to use a variational approximation to $P(T | X, \theta)$ (e.g.,
        a mean field approximation), and minimize the KL divergence between $q$
        and the variational approximation. This approach is known as \textbf{variational EM}.
    \end{summary}
    \vspace{-2mm}
\end{framed}

\subsection*{M-Step Details}
During the M-step we fix $q$ to the value we computed during the E-step and try
to find $\theta^{(k+1)}$ that maximizes $\mathcal{L}$:
\begin{equation*}
    \theta^{(k+1)} = \arg \max_\theta \mathcal{L}(\theta, q^{(k+1)})
\end{equation*}
Here, we decompose the variational lower bound into terms that depend on
$\theta$:
\begin{align*}
    \mathcal{L}(q, \theta) &= \sum_{i=1}^N \sum_{c=1}^T q(t_i = c) \log \left[
        \frac{P(x_i, t_i = c | \theta)}{q(t_i = c)} \right] \\
    &= \sum_{i=1}^N \sum_{c=1}^T q(t_i = c) \log P(x_i, t_i = c | \theta) -
    \underbrace{\sum_{i=1}^N \sum_{c=1}^T q(t_i = c) \log q(t_i = c)}_{\text{does not depend
    on $\theta$}}\\
    &\propto \sum_{i=1}^N \sum_{c=1}^T q(t_i = c) \log P(x_i, t_i = c |
    \theta)\\
    &\propto \mathbb{E}_q[ \log P(X, T | \theta) ]
\end{align*}
This expectation, $\mathbb{E}_q[ \log P(X, T | \theta) ]$ is usually concave and
tends to be relatively easy to maximize with respect to $\theta$ for most
models.

\begin{framed}
    \vspace{-5mm}
    \begin{summary} \textit{M-Step}\\
        During the \textbf{M-step} we fix $q$ to the value we computed during
        the previous E-step and try to find $\theta^{(k+1)}$ that maximizes
        $\mathcal{L}$:
        \begin{equation*}
            \theta^{(k+1)} = \arg \max_\theta \mathcal{L}(\theta, q^{(k+1)})
        \end{equation*}
        In the derivation above, we showed that this is equivalent to finding
        $\theta$ that maximizes the following expected value:
        \begin{equation}
            \theta^{(k+1)} = \arg \max_\theta \mathbb{E}_q[ \log P(X, T | \theta) ]
        \end{equation}
        This is the  M-step update, typically achieved by taking the partial
        derivative of the above expectation wrt each parameter, setting it to 0,
        and solving.
    \end{summary}
    \vspace{-2mm}
\end{framed}





\end{document}
